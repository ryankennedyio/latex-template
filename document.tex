\documentclass{article}

\usepackage{fancyhdr}
\usepackage{amsmath}
\usepackage{amsfonts}
\usepackage{cool}

%
% Basic Document Settings
%

\topmargin=-0.45in
\evensidemargin=0in
\oddsidemargin=0in
\textwidth=6.5in
\textheight=9.0in
\headsep=0.25in

\linespread{1.1}

\pagestyle{fancy}
\chead{Document Header Here}
\rhead{\firstxmark}
\lfoot{\lastxmark}
\cfoot{\thepage}

\setcounter{secnumdepth}{0}
\setlength\parindent{0pt}

% Useful Commands
\newcommand{\E}{\mathrm{E}}
\newcommand{\deriv}[1]{\frac{\partial #1}{\partial #2}}

\begin{document}

\begin{section}{Problem Description}
  Describe problem here. \\

  \textbf{\large Solution}

  Examples of partial derivatives with the 'cool' package.

  \begin{equation*}
    \pderiv{f}{x} \quad
    \pderiv[2]{f}{x}
  \end{equation*}


\end{section}


\end{document}

\end{document}
